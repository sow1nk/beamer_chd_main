% This is an example of CHD Beamer Template
% 4:3
\documentclass[11pt,aspectratio=43,xcolor={dvipsnames},hyperref={pdftex,pdfpagemode=UseNone,hidelinks,pdfdisplaydoctitle=true},usepdftitle=false]{ctexbeamer}
% 16:9
% \documentclass[11pt,aspectratio=169,xcolor={dvipsnames},hyperref={pdftex,pdfpagemode=UseNone,hidelinks,pdfdisplaydoctitle=true},usepdftitle=false]{ctexbeamer}
\usetheme{chd}
\usepackage{lipsum}
\usepackage{ulem} % 下划线
\usefonttheme[onlymath]{serif}  % 仅数学公式使用衬线字体
\usepackage{calligra} % 手写体
\usepackage{tcolorbox}
\usepackage{booktabs} % 三线表格
\usepackage{tabularx} % 自适应宽度表格
\usepackage{array} % 表格列格式扩展
\usepackage{adjustbox} % 表格调整
\usepackage{longtable} % 长表格

\usepackage[colorful,cn]{chdtheoremstyle} % 定理环境 配色和语言选项

\title[CHD Beamer Template]{A brief example in English \\ \normalsize{For CHD Beamer Theme}}
\author{\textbf{Ruixiao Xu}}
\institute{School of Information Engineering\\Chang'an University}
\date{\textbf{\today}}

\begin{document}

\begin{frame}
  \maketitle
\end{frame}

\begin{frame}{\underline{Contents}}
  \tableofcontents
\end{frame}

\section{section 1}
\begin{frame}{\underline{Introduction}}
  \alert{CHD Beamer Template} is an unofficial theme for Chang'an University.
\end{frame}

\begin{frame}{\underline{Sample Page}}
  A matrix $A$ is called normal, if $AA^*=A^*\mskip-2muA$.
\end{frame}

\begin{frame}{\underline{Step-by-Step Display}}
  \begin{itemize}
    \item First point: This is the first item to appear
    \pause
    \item Second point: This is the second item to appear
    \pause  
    \item Third point: This is the third item to appear
    \pause
  \end{itemize}
  
  \vspace{0.5cm}
  
  Now let's show a mathematical formula:
  \pause
  \begin{equation}
    E = mc^2
  \end{equation}
  \pause
  
  Finally, here's an important note:
  \pause
  \begin{alertblock}{Important Tip}
    Use the \texttt{\textbackslash pause} command to control step-by-step display!
  \end{alertblock}
\end{frame}

\section{section 2}
\subsection{Point-by-Point}
\begin{frame}
  \subsectionpage
\end{frame}
\begin{frame}{\underline{Enumerate and itemize}}
  \begin{enumerate}
    \item Hello
    \item There
  \end{enumerate}
  \begin{itemize}
    \item Hello
    \item There
          \begin{itemize}
            \item Subitem
          \end{itemize}
  \end{itemize}
\end{frame}

\begin{frame}{\underline{Blocks}}
  \itshape\lipsum[3]
\end{frame}

\begin{frame}{\underline{Two-Column Image Display}}
  \begin{columns}
    \begin{column}{0.48\textwidth}
      \begin{figure}
        \centering
        \includegraphics[width=0.9\textwidth]{sources/fig1.jpg}
        \caption{色狗}
      \end{figure}
    \end{column}
    
    \hfill
    
    \begin{column}{0.48\textwidth}
      \begin{figure}
        \centering
        \includegraphics[width=0.9\textwidth]{sources/fig2.jpg}
        \caption{略略略}
      \end{figure}
    \end{column}
  \end{columns}
  
  \vspace{0.5cm}
\end{frame}

\begin{frame}{\underline{Three-Line Table}}
  \centering
  
\begin{table}[h]
  \centering
  \caption{Table Caption Example}
  \begin{adjustbox}{width=0.5\textwidth,center}
    \begin{tabular}{c|c|c|c|c}
      \toprule
      \textbf{表头} & \textbf{T1} & \textbf{T2} & \textbf{T3} & \textbf{T4} \\
      \midrule
      base & 50 & 12.34 & 2.15 & - \\
      A & 45 & 15.67 & 2.89 & 0.032 \\
      B & 48 & 18.92 & 3.21 & 0.001 \\
      C & 52 & 21.45 & 2.76 & <0.001 \\
      \bottomrule
    \end{tabular}
  \end{adjustbox}
\end{table} 
  \vspace{0.3cm}
  
  \begin{itemize}
    \item 使用\texttt{\textcolor{blue}{table}}环境包裹表格
    \item \texttt{\textcolor{red}{adjustbox}}宏包动态调整表格大小和位置
    \item \texttt{\textcolor{green}{tabular}}环境创建表格结构
  \end{itemize}
\end{frame}

\subsection{Blocks Comparison}
\begin{frame}
  \subsectionpage
\end{frame}
\begin{frame}{\underline{Blocks}}
  \begin{block}{Example}
    This is a normal block in beamer.
  \end{block}

  \begin{exampleblock}{Example}
    This is an example block in beamer.
  \end{exampleblock}

  \begin{alertblock}{Example}
    This is an alert block in beamer.
  \end{alertblock}

  \begin{tcolorbox}
    This is a tcolorbox.
  \end{tcolorbox}

  \begin{tcolorbox}[colframe=purple!70,colback=red!10]
    This is a tcolorbox with red frame.
  \end{tcolorbox}
\end{frame}

\begin{frame}{\underline{Block Comparison - Alert}}
  % 自定义alert环境
  \begin{chdalert}{重要警告}
    使用本模板时请注意:
    \begin{itemize}
      \item 与\textcolor{red}{remark block一致}
    \end{itemize}
  \end{chdalert}
  
  \vspace{0.3cm}
  
  % 标准Beamer alertblock
  \begin{alertblock}{Standard Beamer Alert}
    这是标准的Beamer \texttt{alertblock},用于对比。
    \begin{itemize}
      \item $\dots$
    \end{itemize}
  \end{alertblock}
\end{frame}


\begin{frame}{\underline{Block Comparison - Example}}
  % 自定义example环境
  \begin{chdexample}{求解二次方程}
    求解方程 $x^2 - 5x + 6 = 0$ 的根。
    
    解:使用求根公式,$x = \frac{5 \pm \sqrt{25-24}}{2} = \frac{5 \pm 1}{2}$
    
    因此 $x_1 = 3, x_2 = 2$。
  \end{chdexample}
  
  \vspace{0.3cm}
  
  % 标准Beamer exampleblock
  \begin{exampleblock}{Standard Beamer Example}
    这是标准的Beamer \texttt{exampleblock}。
    
    同样的例子:$x^2 - 5x + 6 = 0$
    
    解:$x_1 = 3, x_2 = 2$
  \end{exampleblock}
\end{frame}

\begin{frame}{\underline{Block Comparison - Theorem}}
  % 自定义theorem环境
  \begin{chdtheorem}{勾股定理}
    在直角三角形中,直角边的平方和等于斜边的平方。
    
    即:$a^2 + b^2 = c^2$
    
    其中 $c$ 为斜边,$a, b$ 为直角边。
  \end{chdtheorem}
  
  \vspace{0.3cm}
  
  % 标准Beamer block(因为没有theorem block)
  \begin{block}{Standard Block (as Theorem)}
    标准Beamer没有专门的定理块,所以使用普通的\texttt{block}。
    
    即:$a^2 + b^2 = c^2$
    
    其中 $c$ 为斜边,$a, b$ 为直角边。
  \end{block}
\end{frame}

\begin{frame}{\underline{Block Comparison - Definition}}
  % 自定义definition环境
  \begin{chddefinition}{连续函数}
    函数 $f: \mathbb{R} \to \mathbb{R}$ 在点 $x_0$ 连续当且仅当:
    $$\lim_{x \to x_0} f(x) = f(x_0)$$
  \end{chddefinition}
  
  \vspace{0.3cm}
  
  % 标准Beamer block作为定义
  \begin{block}{Standard Block (as Definition)}
    使用标准\texttt{block}来表示定义:
    
    函数 $f: \mathbb{R} \to \mathbb{R}$ 在点 $x_0$ 连续当且仅当:
    $$\lim_{x \to x_0} f(x) = f(x_0)$$
  \end{block}
\end{frame}

\begin{frame}{\underline{Block Comparison - Property}}
  % 自定义property环境
  \begin{chdproperty}{实数的性质}
    对于任意实数 $a, b, c$,有以下性质:
    \begin{itemize}
      \item 交换律:$a + b = b + a$
      \item 结合律:$(a + b) + c = a + (b + c)$
      \item 分配律:$a(b + c) = ab + ac$
    \end{itemize}
  \end{chdproperty}
  
  \vspace{0.3cm}
  
  % 标准Beamer block作为性质
  \begin{block}{Standard Block (as Property)}
    使用标准\texttt{block}来表示性质:
    \begin{itemize}
      \item 交换律:$a + b = b + a$
      \item 结合律:$(a + b) + c = a + (b + c)$
      \item 分配律:$a(b + c) = ab + ac$
    \end{itemize}
  \end{block}
\end{frame}

\begin{frame}{\underline{Block Comparison - Proposition}}
  % 自定义proposition环境
  \begin{chdproposition}{素数的无穷性}
    存在无穷多个素数。
    
    \textbf{简要说明:}假设只有有限个素数 $p_1, p_2, \ldots, p_n$,
    考虑数 $N = p_1 p_2 \cdots p_n + 1$,则 $N$ 不被任何 $p_i$ 整除,
    因此 $N$ 要么是素数,要么有新的素数因子。
  \end{chdproposition}
  
  \vspace{0.3cm}
  
  % 标准Beamer block作为命题
  \begin{block}{Standard Block (as Proposition)}
    使用标准\texttt{block}来表示命题:
    
    存在无穷多个素数。(证明略)
  \end{block}
\end{frame}

\begin{frame}{\underline{Block Comparison - Condition}}
  % 自定义condition环境
  \begin{chdcondition}{连续函数的条件}
    函数 $f: \mathbb{R} \to \mathbb{R}$ 在点 $x_0$ 连续当且仅当:
    $$\lim_{x \to x_0} f(x) = f(x_0)$$
    
    等价地,对于任意 $\varepsilon > 0$,存在 $\delta > 0$ 使得:
    $$|x - x_0| < \delta \Rightarrow |f(x) - f(x_0)| < \varepsilon$$
  \end{chdcondition}
  
  \vspace{0.3cm}
  
  % 标准Beamer block作为条件
  \begin{block}{Standard Block (as Condition)}
    使用标准\texttt{block}来表示条件:
    
    连续性条件:$\lim_{x \to x_0} f(x) = f(x_0)$
  \end{block}
\end{frame}

\begin{frame}{\underline{Block Comparison - Lemma}}
  % 自定义lemma环境
  \begin{chdlemma}{Zorn引理}
    若偏序集合的每个全序子集都有上界,则该偏序集合有极大元。
    
    \textbf{应用:}Zorn引理在抽象代数和泛函分析中有重要应用,
    如证明每个向量空间都有基,每个环都有极大理想等。
  \end{chdlemma}
  
  \vspace{0.3cm}
  
  % 标准Beamer block作为引理
  \begin{block}{Standard Block (as Lemma)}
    使用标准\texttt{block}来表示引理:
    
    Zorn引理:若偏序集合的每个全序子集都有上界,则该偏序集合有极大元。
  \end{block}
\end{frame}

\begin{frame}{\underline{Block Comparison - Algorithm}}
  % 自定义algorithm环境
  \begin{chdalgorithm}{冒泡排序}
    \textbf{输入:}数组 $A[1...n]$
    
    \textbf{步骤:}
    \begin{enumerate}
      \item for $i = 1$ to $n-1$ do
      \item \quad for $j = 1$ to $n-i$ do
      \item \quad\quad if $A[j] > A[j+1]$ then swap($A[j], A[j+1]$)
    \end{enumerate}
    
    \textbf{时间复杂度:}$O(n^2)$
  \end{chdalgorithm}
  
  \vspace{0.2cm}
  
  % 使用exampleblock展示算法
  \begin{exampleblock}{Standard Beamer Algorithm}
    标准Beamer使用\texttt{exampleblock}来展示算法:
    $\dots$
  \end{exampleblock}
\end{frame}

\begin{frame}{\underline{Block Comparison - Proof}}
  % 自定义proof环境
  \begin{chdproof}{勾股定理的证明}
    考虑边长为 $a+b$ 的正方形,可以用两种方式计算其面积。
    
    第一种:$(a+b)^2 = a^2 + 2ab + b^2$
    
    第二种:大正方形面积等于中间正方形加上四个直角三角形的面积:
    $c^2 + 4 \times \frac{1}{2}ab = c^2 + 2ab$
    
    因此:$a^2 + 2ab + b^2 = c^2 + 2ab$,得 $a^2 + b^2 = c^2$
  \end{chdproof}
  
  \vspace{0.2cm}
  
  % 标准block作为证明
  \begin{block}{Standard Block (as Proof)}
    标准Beamer使用普通的\texttt{block}来表示证明。
    
    考虑边长为 $a+b$ 的正方形,可以用两种方式计算其面积...
    
    因此得证:$a^2 + b^2 = c^2$
  \end{block}
\end{frame}

\begin{frame}
  \begin{center}
    \Huge\calligra{Thanks!}
  \end{center}
\end{frame}

\end{document}