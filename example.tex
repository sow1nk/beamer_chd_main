% This is an example of CHD Beamer Template
% 4:3
\documentclass[11pt,aspectratio=43,xcolor={dvipsnames},hyperref={pdftex,pdfpagemode=UseNone,hidelinks,pdfdisplaydoctitle=true},usepdftitle=false]{ctexbeamer}
% 16:9
% \documentclass[11pt,aspectratio=169,xcolor={dvipsnames},hyperref={pdftex,pdfpagemode=UseNone,hidelinks,pdfdisplaydoctitle=true},usepdftitle=false]{ctexbeamer}
\usetheme{chd}
\usepackage{lipsum}
\usepackage{ulem} % 下划线
\usefonttheme[onlymath]{serif}  % 仅数学公式使用衬线字体
\usepackage{calligra} % 手写体
\usepackage{tcolorbox}
\usepackage{booktabs} % 三线表格
\usepackage{tabularx} % 自适应宽度表格
\usepackage{array} % 表格列格式扩展
\usepackage{adjustbox} % 表格调整
\usepackage{longtable} % 长表格

\usepackage[colorful,cn]{chdtheoremstyle} % 定理环境 配色和语言选项

\title[CHD Beamer Template]{A brief example in English \\ \normalsize{For CHD Beamer Theme}}
\author{\textbf{Ruixiao Xu}}
\institute{School of Information Engineering\\Chang'an University}
\date{\textbf{\today}}

\begin{document}

\begin{frame}
  \maketitle
\end{frame}

\begin{frame}{\underline{Contents}}
  \tableofcontents
\end{frame}

\section{section 1}
\begin{frame}{\underline{Introduction}}
  \alert{CHD Beamer Template} is an unofficial theme for Chang'an University.
\end{frame}

\begin{frame}{\underline{Sample Page}}
  A matrix $A$ is called normal, if $AA^*=A^*\mskip-2muA$.
\end{frame}

\section{section 2}
\subsection{section 2.2}
\begin{frame}{\underline{Theorem environments}}

  \begin{theorem}[Sample theorem]
    Hello, there
  \end{theorem}

  \begin{proof}
    Let $x \in \mathbb{R}$. Assume $x > 0$. Then, we have
    \begin{equation}
      x^2 > 0.
    \end{equation}
    Proved.
  \end{proof}
\end{frame}

\begin{frame}{\underline{Enumerate and itemize}}
  \begin{enumerate}
    \item Hello
    \item There
  \end{enumerate}
  \begin{itemize}
    \item Hello
    \item There
          \begin{itemize}
            \item Subitem
          \end{itemize}
  \end{itemize}
\end{frame}

\begin{frame}{\underline{Blocks}}
  \itshape\lipsum[3]
\end{frame}

\begin{frame}{\underline{Blocks}}
  \begin{block}{Example}
    This is a normal block in beamer.
  \end{block}

  \begin{exampleblock}{Example}
    This is an example block in beamer.
  \end{exampleblock}

  \begin{alertblock}{Example}
    This is an alert block in beamer.
  \end{alertblock}

  \begin{tcolorbox}
    This is a tcolorbox.
  \end{tcolorbox}

  \begin{tcolorbox}[colframe=purple!70,colback=red!10]
    This is a tcolorbox with red frame.
  \end{tcolorbox}
\end{frame}

\begin{frame}{\underline{Two-Column Image Display}}
  \begin{columns}
    \begin{column}{0.48\textwidth}
      \begin{figure}
        \centering
        \includegraphics[width=0.9\textwidth]{sources/fig1.jpg}
        \caption{色狗}
      \end{figure}
    \end{column}
    
    \hfill
    
    \begin{column}{0.48\textwidth}
      \begin{figure}
        \centering
        \includegraphics[width=0.9\textwidth]{sources/fig2.jpg}
        \caption{略略略}
      \end{figure}
    \end{column}
  \end{columns}
  
  \vspace{0.5cm}
\end{frame}

\begin{frame}{\underline{Three-Line Table Demo}}
  \centering
  
\begin{table}[h]
  \centering
  \caption{Table Caption Example}
  \begin{adjustbox}{width=0.5\textwidth,center}
    \begin{tabular}{c|c|c|c|c}
      \toprule
      \textbf{表头} & \textbf{T1} & \textbf{T2} & \textbf{T3} & \textbf{T4} \\
      \midrule
      base & 50 & 12.34 & 2.15 & - \\
      A & 45 & 15.67 & 2.89 & 0.032 \\
      B & 48 & 18.92 & 3.21 & 0.001 \\
      C & 52 & 21.45 & 2.76 & <0.001 \\
      \bottomrule
    \end{tabular}
  \end{adjustbox}
\end{table} 
  \vspace{0.3cm}
  
  \begin{itemize}
    \item 使用\texttt{\textcolor{blue}{table}}环境包裹表格
    \item \texttt{\textcolor{red}{adjustbox}}宏包动态调整表格大小和位置
    \item \texttt{\textcolor{green}{tabular}}环境创建表格结构
  \end{itemize}
\end{frame}

\begin{frame}{\underline{Theorem Environments}}
  % 例子环境 - 用于展示具体实例
  \begin{chdexample}{求解二次方程}
    求解方程 $x^2 - 5x + 6 = 0$ 的根。
    
    解:使用求根公式,$x = \frac{5 \pm \sqrt{25-24}}{2} = \frac{5 \pm 1}{2}$
    
    因此 $x_1 = 3, x_2 = 2$。
  \end{chdexample}
  % 定理环境 - 用于重要的数学定理
  \begin{chdtheorem}{勾股定理}
    在直角三角形中,直角边的平方和等于斜边的平方。
    
    即:$a^2 + b^2 = c^2$
    
    其中 $c$ 为斜边,$a, b$ 为直角边。
  \end{chdtheorem}
\end{frame}

\begin{frame}{\underline{Theorem Environments}}
  % 性质环境 - 用于描述数学性质
  \begin{chdproperty}{实数的性质}
    对于任意实数 $a, b, c$,有以下性质:
    \begin{itemize}
      \item 交换律:$a + b = b + a$
      \item 结合律:$(a + b) + c = a + (b + c)$
      \item 分配律:$a(b + c) = ab + ac$
    \end{itemize}
  \end{chdproperty}
  % 命题环境 - 用于数学命题
  \begin{chdproposition}{素数的无穷性}
    存在无穷多个素数。
  \end{chdproposition}
\end{frame}

\begin{frame}{\underline{Theorem Environments}}
  % 算法环境 - 用于描述算法步骤
  \begin{chdalgorithm}{冒泡排序}
    \textbf{输入:}数组 $A[1...n]$
    
    \textbf{步骤:}
    \begin{enumerate}
      \item for $i = 1$ to $n-1$ do
      \item \quad for $j = 1$ to $n-i$ do
      \item \quad\quad if $A[j] > A[j+1]$ then swap($A[j], A[j+1]$)
    \end{enumerate}
    
    \textbf{时间复杂度:}$O(n^2)$
  \end{chdalgorithm}
  % 公理环境 - 用于数学公理
  \begin{chdaxiom}{选择公理}
    对于任意非空集合的集合 $\mathcal{F}$,存在选择函数 $f$,使得对于 $\mathcal{F}$ 中每个非空集合 $S$,都有 $f(S) \in S$。
  \end{chdaxiom}
\end{frame}

\begin{frame}{\underline{Theorem Environments}}
  % 条件环境 - 用于描述数学条件
  \begin{chdcondition}{连续函数的条件}
    函数 $f: \mathbb{R} \to \mathbb{R}$ 在点 $x_0$ 连续当且仅当:
    $$\lim_{x \to x_0} f(x) = f(x_0)$$
  \end{chdcondition}
  % 引理环境 - 用于辅助证明
  \begin{chdlemma}{Zorn引理}
    若偏序集合的每个全序子集都有上界,则该偏序集合有极大元。
  \end{chdlemma}
\end{frame}

\begin{frame}{\underline{Theorem Environments}}
  % 注释环境 - 用于补充说明
  \begin{chdremark}{关于定理环境的说明}
    本模板提供了丰富的定理环境,支持自动编号和交叉引用。
  \end{chdremark}
  % 证明环境 - 用于数学证明
  \begin{chdproof}{勾股定理的证明}
    考虑边长为 $a+b$ 的正方形,可以用两种方式计算其面积...
  \end{chdproof}
\end{frame}

\begin{frame}
  \begin{center}
    \Huge\calligra{Thanks!}
  \end{center}
\end{frame}

\end{document}