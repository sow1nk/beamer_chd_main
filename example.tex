% This is an example of CHD Beamer Template
% 4:3
\documentclass[11pt,aspectratio=43,xcolor={dvipsnames},hyperref={pdftex,pdfpagemode=UseNone,hidelinks,pdfdisplaydoctitle=true},usepdftitle=false]{ctexbeamer}
% 16:9
% \documentclass[11pt,aspectratio=169,xcolor={dvipsnames},hyperref={pdftex,pdfpagemode=UseNone,hidelinks,pdfdisplaydoctitle=true},usepdftitle=false]{ctexbeamer}
\usetheme{chd}
\usepackage{lipsum}
\usepackage{ulem} % 下划线
\usefonttheme[onlymath]{serif}  % 仅数学公式使用衬线字体
\usepackage{calligra} % 手写体
\usepackage{tcolorbox}
\usepackage{booktabs} % 三线表格
\usepackage{tabularx} % 自适应宽度表格
\usepackage{array} % 表格列格式扩展
\usepackage{adjustbox} % 表格调整
\usepackage{longtable} % 长表格

\usepackage[colorful,cn]{chdtheoremstyle} % 定理环境 配色和语言选项

\title{CHD Beamer Template}
\author{\textbf{Ruixiao Xu}}
\institute{School of Information Engineering\\Chang'an University}
\date{\today}

\begin{document}

\begin{frame}
  \maketitle
\end{frame}

\begin{frame}{\underline{Contents}}
  \tableofcontents
\end{frame}

\section{section 1}
\begin{frame}{\underline{Introduction}}
  \alert{CHD Beamer Template} is an unofficial theme for Chang'an University.
\end{frame}

\begin{frame}{\underline{Sample Page}}
  A matrix $A$ is called normal, if $AA^*=A^*\mskip-2muA$.
\end{frame}

\section{section 2}
\subsection{section 2.2}
\begin{frame}{\underline{Theorem environments}}

  \begin{theorem}[Sample theorem]
    Hello, there
  \end{theorem}

  \begin{proof}
    Let $x \in \mathbb{R}$. Assume $x > 0$. Then, we have
    \begin{equation}
      x^2 > 0.
    \end{equation}
    Proved.
  \end{proof}
\end{frame}

\begin{frame}{\underline{Enumerate and itemize}}
  \begin{enumerate}
    \item Hello
    \item There
  \end{enumerate}
  \begin{itemize}
    \item Hello
    \item There
          \begin{itemize}
            \item Subitem
          \end{itemize}
  \end{itemize}
\end{frame}

\begin{frame}{\underline{Blocks}}
  \itshape\lipsum[3]
\end{frame}

\begin{frame}{\underline{Blocks}}
  \begin{block}{Example}
    This is a normal block in beamer.
  \end{block}

  \begin{exampleblock}{Example}
    This is an example block in beamer.
  \end{exampleblock}

  \begin{alertblock}{Example}
    This is an alert block in beamer.
  \end{alertblock}

  \begin{tcolorbox}
    This is a tcolorbox.
  \end{tcolorbox}

  \begin{tcolorbox}[colframe=purple!70,colback=red!10]
    This is a tcolorbox with red frame.
  \end{tcolorbox}
\end{frame}

\begin{frame}{\underline{Two-Column Image Display}}
  \begin{columns}
    \begin{column}{0.48\textwidth}
      \begin{figure}
        \centering
        \includegraphics[width=0.9\textwidth]{sources/fig1.jpg}
        \caption{色狗}
      \end{figure}
    \end{column}
    
    \hfill
    
    \begin{column}{0.48\textwidth}
      \begin{figure}
        \centering
        \includegraphics[width=0.9\textwidth]{sources/fig2.jpg}
        \caption{略略略}
      \end{figure}
    \end{column}
  \end{columns}
  
  \vspace{0.5cm}
\end{frame}

\begin{frame}{\underline{Three-Line Table Demo}}
  \centering
  
\begin{table}[h]
  \centering
  \caption{Table Caption Example}
  \begin{adjustbox}{width=0.5\textwidth,center}
    \begin{tabular}{c|c|c|c|c}
      \toprule
      \textbf{表头} & \textbf{T1} & \textbf{T2} & \textbf{T3} & \textbf{T4} \\
      \midrule
      base & 50 & 12.34 & 2.15 & - \\
      A & 45 & 15.67 & 2.89 & 0.032 \\
      B & 48 & 18.92 & 3.21 & 0.001 \\
      C & 52 & 21.45 & 2.76 & <0.001 \\
      \bottomrule
    \end{tabular}
  \end{adjustbox}
\end{table} 
  \vspace{0.3cm}
  
  \begin{itemize}
    \item 使用\texttt{\textcolor{blue}{table}}环境包裹表格
    \item \texttt{\textcolor{red}{adjustbox}}宏包动态调整表格大小和位置
    \item \texttt{\textcolor{green}{tabular}}环境创建表格结构
  \end{itemize}
\end{frame}

\begin{frame}{\underline{CHD Theorem Environments}}
  % 例子环境 - 用于展示具体实例
  \begin{chdexample}{chdexample 环境示例}
    这是一个 \texttt{chdexample} 环境的示例。
  \end{chdexample}
  % 定理环境 - 用于重要的数学定理
  \begin{chdtheorem}{chdtheorem 环境定理}
    这是一个 \texttt{chdtheorem} 环境的示例。
  \end{chdtheorem}
  % 注释环境 - 用于补充说明
  \begin{chdremark}{chdremark 环境注释}
    这是一个 \texttt{chdremark} 环境的示例。
  \end{chdremark}
  % 证明环境 - 用于数学证明
  \begin{chdproof}{chdproof 环境证明}
    这是一个 \texttt{chdproof} 环境的示例。
  \end{chdproof}
\end{frame}

\begin{frame}{\underline{CHD Theorem Environments}}
  % 性质环境 - 用于描述数学性质
  \begin{chdproperty}{chdproperty 环境性质}
    这是一个 \texttt{chdproperty} 环境的示例。
  \end{chdproperty}
  % 命题环境 - 用于数学命题
  \begin{chdproposition}{chdproposition 环境命题}
    这是一个 \texttt{chdproposition} 环境的示例。
  \end{chdproposition}
  % 算法环境 - 用于描述算法步骤
  \begin{chdalgorithm}{chdalgorithm 环境算法}
    这是一个 \texttt{chdalgorithm} 环境的示例。
  \end{chdalgorithm}
  % 公理环境 - 用于数学公理
  \begin{chdaxiom}{chdaxiom 环境公理}
    这是一个 \texttt{chdaxiom} 环境的示例。
  \end{chdaxiom}
\end{frame}

\begin{frame}{\underline{CHD Theorem Environments}}
  % 条件环境 - 用于描述数学条件
  \begin{chdcondition}{chdcondition 环境条件}
    这是一个 \texttt{chdcondition} 环境的示例。
  \end{chdcondition}
  % 引理环境 - 用于辅助证明
  \begin{chdlemma}{chdlemma 环境引理}
    这是一个 \texttt{chdlemma} 环境的示例。
  \end{chdlemma}
\end{frame}

\begin{frame}
  \begin{center}
    \Huge\calligra{Thanks!}
  \end{center}
\end{frame}

\end{document}